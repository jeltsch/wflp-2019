\documentclass{llncs}

\usepackage{isabelle}
\usepackage{isabellesym}
\usepackage[hyphens]{url}
\usepackage{pdfsetup}

\usepackage{microtype}

\usepackage{latexsym}
\usepackage{amssymb}

\usepackage{amsmath}

\renewenvironment{isamarkuptext}{\par}{}
\renewenvironment{isamarkuptxt}{\par}{}

\renewenvironment{isabellebody}{}{}
\renewenvironment{isabellebodytt}{}{}

\renewenvironment{isabelle}
    {\begin{center}\isaspacing\isastyle}
    {\end{center}}
\renewenvironment{isabellett}
    {\begin{center}\isaspacing\isastylett}
    {\end{center}}

\setlength{\partopsep}{0mm}
%NOTE:
%
%  * Isabelle generates paragraph breaks around list environments, resulting in extra vertical space
%    above lists. Setting \partopsep to zero prevents this extra space from being added.
%
%  * That said, that automatic adding of paragraph breaks causes text after a list to be considered
%    forming a new paragraph. If this is not desired, \noindent should be used to prevent the
%    indentation that would signal the start of a new paragraph.

\urlstyle{rm}
\isabellestyle{it}

\begin{document}

\title{A Process Calculus\\for Formally Verifying\\Blockchain Consensus Protocols}
\titlerunning{A Process Calculus for Blockchain Consensus Protocols}
\author{Wolfgang Jeltsch\inst{1,2}\orcidID{0000-0002-8068-8401}}
\institute{Well-Typed\and IOHK\\\email{wolfgang@well-typed.com}}
\maketitle

\begin{abstract}

Blockchains are becoming increasingly relevant in a variety of fields, such as finance, logistics,
and real estate. The fundamental task of a blockchain system is to establish data consistency among
distributed agents in an open network. Blockchain consensus protocols are central for performing
this task.

Since consensus protocols play such a crucial role in blockchain technology, several projects are
underway that apply formal methods to these protocols. One such project is carried out by a team of
the Formal Methods Group at IOHK. This project, in which I am involved, aims at a formally verified
implementation of the Ouroboros family of consensus protocols, the backbone of the Cardano
blockchain. The first outcome of our project is the $\natural$-calculus (pronounced ``natural
calculus''), a general-purpose process calculus that serves as our implementation language. The
$\natural$-calculus is a domain-specific language embedded in a functional host language using
higher-order abstract syntax.

This paper will be a ramble through the $\natural$-calculus. First we will look at its language and
its operational semantics. The latter is unique in that it uses a stack of two labeled transition
systems to treat phenomena like data transfer and the opening and closing of channel scope in a
modular fashion. The presence of multiple transition systems calls for a generic treatment of
derived concurrency concepts such as strong and weak bisimilarity. I will show you how such a
treatment can be achieved by capturing notions like scope opening and silent transitions abstractly
using axiomatically defined algebraic structures. You will, perhaps surprisingly, encounter some old
friends of the functional programmer, namely functors and monads, albeit in a different guise.

\keywords{%
    Blockchain                    \and
    Distributed computing         \and
    Formal verification           \and
    Process calculus              \and
    Functional programming        \and
    Higher-order abstract syntax%
}

\end{abstract}

\input{session}

\bibliographystyle{splncs04}
\bibliography{root}

\end{document}
