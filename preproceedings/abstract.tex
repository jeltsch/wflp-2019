\documentclass{llncs}

\usepackage{microtype}

\usepackage{latexsym}
\usepackage{amssymb}

\usepackage{amsmath}

\begin{document}

\title{%
    A Process Calculus\\
    for Formally Verifying\\
    Blockchain Consensus Protocols%
}
\titlerunning{A Process Calculus for Blockchain Consensus Protocols}
\author{Wolfgang Jeltsch\orcidID{0000-0002-8068-8401}}
\institute{Well-Typed LLP\\\email{wolfgang@well-typed.com}}
\maketitle

\begin{abstract}

Blockchains are becoming increasingly relevant in a variety of fields,
such as finance, logistics, and real estate. The fundamental task of a
blockchain system is to establish data consistency among distributed
agents in an open network. Blockchain consensus protocols are central
for performing this task.

Since consensus protocols play such a crucial role in blockchain
technology, several projects are underway that apply formal methods to
these protocols. One such project is carried out by a team of the Formal
Methods Group at IOHK. This project, in which I am involved, aims at a
formally verified implementation of the Ouroboros family of consensus
protocols, the backbone of the Cardano blockchain. The first outcome of
our project is the $\natural$-calculus (pronounced ``natural
calculus''), a general-purpose process calculus that serves as our
implementation language. The $\natural$-calculus is a domain-specific
language embedded in a functional host language using higher-order
abstract syntax.

The syntax of the $\natural$-calculus is rather minimalistic. It
comprises only five constructs: $\mathit{stop}$, $\mathit{send}$,
$\mathit{receive}$, $\mathit{parallel}$, and
$\mathit{new\text{-}channel}$. The operational semantics of the
$\natural$-calculus is unique in that it uses a stack of two labeled
transition systems to treat phenomena like data transfer and the opening
and closing of channel scope in a modular fashion. The presence of
multiple transition systems calls for a generic treatment of derived
concurrency concepts such as strong and weak bisimilarity. We achieve
such a treatment by capturing notions like scope opening and silent
transitions abstractly using axiomatically defined algebraic structures.
At the heart of these structures are, perhaps surprisingly, two old
friends of the functional programmer, namely the functor and the monad,
albeit in a different guise.

\keywords{%
    Blockchain                    \and
    Distributed computing         \and
    Formal verification           \and
    Process calculus              \and
    Functional programming        \and
    Higher-order abstract syntax%
}

\end{abstract}

\end{document}
